% Options for packages loaded elsewhere
\PassOptionsToPackage{unicode}{hyperref}
\PassOptionsToPackage{hyphens}{url}
%
\documentclass[
]{article}
\usepackage{amsmath,amssymb}
\usepackage{lmodern}
\usepackage{ifxetex,ifluatex}
\ifnum 0\ifxetex 1\fi\ifluatex 1\fi=0 % if pdftex
  \usepackage[T1]{fontenc}
  \usepackage[utf8]{inputenc}
  \usepackage{textcomp} % provide euro and other symbols
\else % if luatex or xetex
  \usepackage{unicode-math}
  \defaultfontfeatures{Scale=MatchLowercase}
  \defaultfontfeatures[\rmfamily]{Ligatures=TeX,Scale=1}
\fi
% Use upquote if available, for straight quotes in verbatim environments
\IfFileExists{upquote.sty}{\usepackage{upquote}}{}
\IfFileExists{microtype.sty}{% use microtype if available
  \usepackage[]{microtype}
  \UseMicrotypeSet[protrusion]{basicmath} % disable protrusion for tt fonts
}{}
\makeatletter
\@ifundefined{KOMAClassName}{% if non-KOMA class
  \IfFileExists{parskip.sty}{%
    \usepackage{parskip}
  }{% else
    \setlength{\parindent}{0pt}
    \setlength{\parskip}{6pt plus 2pt minus 1pt}}
}{% if KOMA class
  \KOMAoptions{parskip=half}}
\makeatother
\usepackage{xcolor}
\IfFileExists{xurl.sty}{\usepackage{xurl}}{} % add URL line breaks if available
\IfFileExists{bookmark.sty}{\usepackage{bookmark}}{\usepackage{hyperref}}
\hypersetup{
  pdftitle={SAT study},
  pdfauthor={Tyler Frankenberg},
  hidelinks,
  pdfcreator={LaTeX via pandoc}}
\urlstyle{same} % disable monospaced font for URLs
\usepackage[margin=1in]{geometry}
\usepackage{color}
\usepackage{fancyvrb}
\newcommand{\VerbBar}{|}
\newcommand{\VERB}{\Verb[commandchars=\\\{\}]}
\DefineVerbatimEnvironment{Highlighting}{Verbatim}{commandchars=\\\{\}}
% Add ',fontsize=\small' for more characters per line
\usepackage{framed}
\definecolor{shadecolor}{RGB}{248,248,248}
\newenvironment{Shaded}{\begin{snugshade}}{\end{snugshade}}
\newcommand{\AlertTok}[1]{\textcolor[rgb]{0.94,0.16,0.16}{#1}}
\newcommand{\AnnotationTok}[1]{\textcolor[rgb]{0.56,0.35,0.01}{\textbf{\textit{#1}}}}
\newcommand{\AttributeTok}[1]{\textcolor[rgb]{0.77,0.63,0.00}{#1}}
\newcommand{\BaseNTok}[1]{\textcolor[rgb]{0.00,0.00,0.81}{#1}}
\newcommand{\BuiltInTok}[1]{#1}
\newcommand{\CharTok}[1]{\textcolor[rgb]{0.31,0.60,0.02}{#1}}
\newcommand{\CommentTok}[1]{\textcolor[rgb]{0.56,0.35,0.01}{\textit{#1}}}
\newcommand{\CommentVarTok}[1]{\textcolor[rgb]{0.56,0.35,0.01}{\textbf{\textit{#1}}}}
\newcommand{\ConstantTok}[1]{\textcolor[rgb]{0.00,0.00,0.00}{#1}}
\newcommand{\ControlFlowTok}[1]{\textcolor[rgb]{0.13,0.29,0.53}{\textbf{#1}}}
\newcommand{\DataTypeTok}[1]{\textcolor[rgb]{0.13,0.29,0.53}{#1}}
\newcommand{\DecValTok}[1]{\textcolor[rgb]{0.00,0.00,0.81}{#1}}
\newcommand{\DocumentationTok}[1]{\textcolor[rgb]{0.56,0.35,0.01}{\textbf{\textit{#1}}}}
\newcommand{\ErrorTok}[1]{\textcolor[rgb]{0.64,0.00,0.00}{\textbf{#1}}}
\newcommand{\ExtensionTok}[1]{#1}
\newcommand{\FloatTok}[1]{\textcolor[rgb]{0.00,0.00,0.81}{#1}}
\newcommand{\FunctionTok}[1]{\textcolor[rgb]{0.00,0.00,0.00}{#1}}
\newcommand{\ImportTok}[1]{#1}
\newcommand{\InformationTok}[1]{\textcolor[rgb]{0.56,0.35,0.01}{\textbf{\textit{#1}}}}
\newcommand{\KeywordTok}[1]{\textcolor[rgb]{0.13,0.29,0.53}{\textbf{#1}}}
\newcommand{\NormalTok}[1]{#1}
\newcommand{\OperatorTok}[1]{\textcolor[rgb]{0.81,0.36,0.00}{\textbf{#1}}}
\newcommand{\OtherTok}[1]{\textcolor[rgb]{0.56,0.35,0.01}{#1}}
\newcommand{\PreprocessorTok}[1]{\textcolor[rgb]{0.56,0.35,0.01}{\textit{#1}}}
\newcommand{\RegionMarkerTok}[1]{#1}
\newcommand{\SpecialCharTok}[1]{\textcolor[rgb]{0.00,0.00,0.00}{#1}}
\newcommand{\SpecialStringTok}[1]{\textcolor[rgb]{0.31,0.60,0.02}{#1}}
\newcommand{\StringTok}[1]{\textcolor[rgb]{0.31,0.60,0.02}{#1}}
\newcommand{\VariableTok}[1]{\textcolor[rgb]{0.00,0.00,0.00}{#1}}
\newcommand{\VerbatimStringTok}[1]{\textcolor[rgb]{0.31,0.60,0.02}{#1}}
\newcommand{\WarningTok}[1]{\textcolor[rgb]{0.56,0.35,0.01}{\textbf{\textit{#1}}}}
\usepackage{graphicx}
\makeatletter
\def\maxwidth{\ifdim\Gin@nat@width>\linewidth\linewidth\else\Gin@nat@width\fi}
\def\maxheight{\ifdim\Gin@nat@height>\textheight\textheight\else\Gin@nat@height\fi}
\makeatother
% Scale images if necessary, so that they will not overflow the page
% margins by default, and it is still possible to overwrite the defaults
% using explicit options in \includegraphics[width, height, ...]{}
\setkeys{Gin}{width=\maxwidth,height=\maxheight,keepaspectratio}
% Set default figure placement to htbp
\makeatletter
\def\fps@figure{htbp}
\makeatother
\setlength{\emergencystretch}{3em} % prevent overfull lines
\providecommand{\tightlist}{%
  \setlength{\itemsep}{0pt}\setlength{\parskip}{0pt}}
\setcounter{secnumdepth}{-\maxdimen} % remove section numbering
\ifluatex
  \usepackage{selnolig}  % disable illegal ligatures
\fi

\title{SAT study}
\author{Tyler Frankenberg}
\date{02/27/2022}

\begin{document}
\maketitle

\hypertarget{import-packages}{%
\subsection{Import packages}\label{import-packages}}

\begin{Shaded}
\begin{Highlighting}[]
\FunctionTok{library}\NormalTok{(tidyverse)}
\end{Highlighting}
\end{Shaded}

\hypertarget{import-data}{%
\subsection{Import data}\label{import-data}}

\begin{Shaded}
\begin{Highlighting}[]
\NormalTok{url }\OtherTok{\textless{}{-}} \StringTok{"https://raw.githubusercontent.com/curdferguson/data621/main/datasets/sat.txt"}

\NormalTok{sat }\OtherTok{\textless{}{-}} \FunctionTok{read\_tsv}\NormalTok{(url, }\AttributeTok{skip =} \DecValTok{1}\NormalTok{, }\AttributeTok{col\_names =} \FunctionTok{c}\NormalTok{(}\StringTok{"state"}\NormalTok{, }\StringTok{"expend"}\NormalTok{, }\StringTok{"ratio"}\NormalTok{, }\StringTok{"salary"}\NormalTok{, }\StringTok{"takers"}\NormalTok{, }\StringTok{"verbal"}\NormalTok{, }\StringTok{"math"}\NormalTok{, }\StringTok{"total"}\NormalTok{), }\AttributeTok{show\_col\_types=}\ConstantTok{FALSE}\NormalTok{)}

\NormalTok{sat }\OtherTok{\textless{}{-}} \FunctionTok{column\_to\_rownames}\NormalTok{(sat, }\AttributeTok{var=}\StringTok{"state"}\NormalTok{)}
\end{Highlighting}
\end{Shaded}

\hypertarget{glimpse-dataset-structure-and-each-columns-summary-statistics}{%
\subsection{Glimpse dataset structure and each column's summary
statistics}\label{glimpse-dataset-structure-and-each-columns-summary-statistics}}

\begin{Shaded}
\begin{Highlighting}[]
\NormalTok{sat }\SpecialCharTok{\%\textgreater{}\%} \FunctionTok{head}\NormalTok{(}\DecValTok{5}\NormalTok{)}
\end{Highlighting}
\end{Shaded}

\begin{verbatim}
##            expend ratio salary takers verbal math total
## Alabama     4.405  17.2 31.144      8    491  538  1029
## Alaska      8.963  17.6 47.951     47    445  489   934
## Arizona     4.778  19.3 32.175     27    448  496   944
## Arkansas    4.459  17.1 28.934      6    482  523  1005
## California  4.992  24.0 41.078     45    417  485   902
\end{verbatim}

\begin{Shaded}
\begin{Highlighting}[]
\NormalTok{sat[,}\DecValTok{1}\SpecialCharTok{:}\DecValTok{4}\NormalTok{] }\SpecialCharTok{\%\textgreater{}\%} \FunctionTok{summary}\NormalTok{()}
\end{Highlighting}
\end{Shaded}

\begin{verbatim}
##      expend          ratio           salary          takers     
##  Min.   :3.656   Min.   :13.80   Min.   :25.99   Min.   : 4.00  
##  1st Qu.:4.882   1st Qu.:15.22   1st Qu.:30.98   1st Qu.: 9.00  
##  Median :5.768   Median :16.60   Median :33.29   Median :28.00  
##  Mean   :5.905   Mean   :16.86   Mean   :34.83   Mean   :35.24  
##  3rd Qu.:6.434   3rd Qu.:17.57   3rd Qu.:38.55   3rd Qu.:63.00  
##  Max.   :9.774   Max.   :24.30   Max.   :50.05   Max.   :81.00
\end{verbatim}

\begin{Shaded}
\begin{Highlighting}[]
\FunctionTok{cat}\NormalTok{(}\StringTok{"}\SpecialCharTok{\textbackslash{}n}\StringTok{"}\NormalTok{)}
\end{Highlighting}
\end{Shaded}

\begin{Shaded}
\begin{Highlighting}[]
\NormalTok{sat[,}\DecValTok{5}\SpecialCharTok{:}\DecValTok{7}\NormalTok{] }\SpecialCharTok{\%\textgreater{}\%} \FunctionTok{summary}\NormalTok{()}
\end{Highlighting}
\end{Shaded}

\begin{verbatim}
##      verbal           math           total       
##  Min.   :401.0   Min.   :443.0   Min.   : 844.0  
##  1st Qu.:427.2   1st Qu.:474.8   1st Qu.: 897.2  
##  Median :448.0   Median :497.5   Median : 945.5  
##  Mean   :457.1   Mean   :508.8   Mean   : 965.9  
##  3rd Qu.:490.2   3rd Qu.:539.5   3rd Qu.:1032.0  
##  Max.   :516.0   Max.   :592.0   Max.   :1107.0
\end{verbatim}

\hypertarget{anova}{%
\subsection{ANOVA}\label{anova}}

We construct a linear model with \texttt{expend}, \texttt{ratio}, and
\texttt{salary} as predictors of the response variable \texttt{total}.

Is the effect of these predictors on the response statistically
significant? We use an F-test for Analysis of Variance to test whether
any of the predictors' coefficients is statistically different from
zero.

Our F-statistic of 4.0662 is sufficiently greater than that of the null
model, and our p-value of 0.01209 indicates this result would be the
result of chance in only 0.12\% of hypothetical samples.

We reject the null hypothesis that the coefficients of our predictors
are statistically equivalent to zero, and take the effect of this model
on the response as statistically significant at the 0.95 level.

\begin{Shaded}
\begin{Highlighting}[]
\NormalTok{sat\_lm1 }\OtherTok{\textless{}{-}} \FunctionTok{lm}\NormalTok{(total }\SpecialCharTok{\textasciitilde{}}\NormalTok{ expend }\SpecialCharTok{+}\NormalTok{ ratio }\SpecialCharTok{+}\NormalTok{ salary, }\AttributeTok{data=}\NormalTok{sat)}
\NormalTok{lm1\_sum }\OtherTok{\textless{}{-}} \FunctionTok{summary}\NormalTok{(sat\_lm1)}
\NormalTok{lm1\_sum}
\end{Highlighting}
\end{Shaded}

\begin{verbatim}
## 
## Call:
## lm(formula = total ~ expend + ratio + salary, data = sat)
## 
## Residuals:
##      Min       1Q   Median       3Q      Max 
## -140.911  -46.740   -7.535   47.966  123.329 
## 
## Coefficients:
##             Estimate Std. Error t value Pr(>|t|)    
## (Intercept) 1069.234    110.925   9.639 1.29e-12 ***
## expend        16.469     22.050   0.747   0.4589    
## ratio          6.330      6.542   0.968   0.3383    
## salary        -8.823      4.697  -1.878   0.0667 .  
## ---
## Signif. codes:  0 '***' 0.001 '**' 0.01 '*' 0.05 '.' 0.1 ' ' 1
## 
## Residual standard error: 68.65 on 46 degrees of freedom
## Multiple R-squared:  0.2096, Adjusted R-squared:  0.1581 
## F-statistic: 4.066 on 3 and 46 DF,  p-value: 0.01209
\end{verbatim}

\begin{Shaded}
\begin{Highlighting}[]
\NormalTok{sat\_nullmod }\OtherTok{\textless{}{-}} \FunctionTok{lm}\NormalTok{(total }\SpecialCharTok{\textasciitilde{}} \DecValTok{1}\NormalTok{, }\AttributeTok{data=}\NormalTok{sat)}

\NormalTok{lm1\_anova }\OtherTok{\textless{}{-}} \FunctionTok{anova}\NormalTok{(sat\_nullmod, sat\_lm1)}
\NormalTok{lm1\_anova}
\end{Highlighting}
\end{Shaded}

\begin{verbatim}
## Analysis of Variance Table
## 
## Model 1: total ~ 1
## Model 2: total ~ expend + ratio + salary
##   Res.Df    RSS Df Sum of Sq      F  Pr(>F)  
## 1     49 274308                              
## 2     46 216812  3     57496 4.0662 0.01209 *
## ---
## Signif. codes:  0 '***' 0.001 '**' 0.01 '*' 0.05 '.' 0.1 ' ' 1
\end{verbatim}

\hypertarget{examine-the-effect-of-a-new-variable-using-anova-and-t-test}{%
\subsection{Examine the effect of a new variable using ANOVA and
T-test}\label{examine-the-effect-of-a-new-variable-using-anova-and-t-test}}

We add the predictor \texttt{takers} to the model.

Is the additon of this predictor on the response statistically
significant? We can test this in two ways; using a t-test for the
specific variable and using an F-test to compare the effect of the first
and second models as a whole. Then we can show that the results of these
two methods are actually the same.

\hypertarget{t-test}{%
\paragraph{T-test}\label{t-test}}

first, we can output the regresion summary of the new model and observe
the value of the t-statistic and p-value for \texttt{takers}.

Our regression summary output gives a t-value of -12.559 and a p-value
of 2.61e-16 for \texttt{takers}. This indicates the coefficient is about
12.5 times the size of its standard error, and that we'd expect this to
be the result of chance in well fewer than 0.01\% of hypothetical
samples.

We conclude by this result that we can reject the null hypothesis at the
0.95\% level of statistical significance, and assume the impact of
\texttt{takers} to be significant.

\hypertarget{anova-1}{%
\paragraph{ANOVA}\label{anova-1}}

Second, we can use an F-test for Analysis of Variance between the new
model and previous model to test whether the additional impact of the
coefficient for \texttt{takers} is statistically different from zero.

Our F-statistic of 157.74 is sufficiently greater than that of the model
without \texttt{takers}, and our p-value of 2.607e-16 indicates this
result would be the result of chance in well fewer than 0.01\% of
hypothetical samples.

We reject the null hypothesis that the difference in the coefficients of
our predictors is statistically equivalent to zero, and take the effect
of this model on the response as statistically significant at the 0.95
level.

\hypertarget{verify-equivalence}{%
\paragraph{Verify equivalence}\label{verify-equivalence}}

Finally, we can verify that our results from these two tests are the
same. We expect that our ANOVA F statistic should be approximately the
square of our t-value for the added variable \texttt{takers}, and that
their p-values would be equal.

As we see in our output below, the difference between the t-value
squared and the F-statistic, as well as between the p-values, are each
so small as to be functionally equivalent to zero.

\begin{Shaded}
\begin{Highlighting}[]
\CommentTok{\# method 1 {-} regression summary output t{-}test}
\NormalTok{sat\_lm2 }\OtherTok{\textless{}{-}} \FunctionTok{lm}\NormalTok{(total }\SpecialCharTok{\textasciitilde{}}\NormalTok{ expend }\SpecialCharTok{+}\NormalTok{ ratio }\SpecialCharTok{+}\NormalTok{ salary }\SpecialCharTok{+}\NormalTok{ takers, }\AttributeTok{data=}\NormalTok{sat)}
\NormalTok{lm2\_sum }\OtherTok{\textless{}{-}} \FunctionTok{summary}\NormalTok{(sat\_lm2)}
\NormalTok{lm2\_sum}
\end{Highlighting}
\end{Shaded}

\begin{verbatim}
## 
## Call:
## lm(formula = total ~ expend + ratio + salary + takers, data = sat)
## 
## Residuals:
##     Min      1Q  Median      3Q     Max 
## -90.531 -20.855  -1.746  15.979  66.571 
## 
## Coefficients:
##              Estimate Std. Error t value Pr(>|t|)    
## (Intercept) 1045.9715    52.8698  19.784  < 2e-16 ***
## expend         4.4626    10.5465   0.423    0.674    
## ratio         -3.6242     3.2154  -1.127    0.266    
## salary         1.6379     2.3872   0.686    0.496    
## takers        -2.9045     0.2313 -12.559 2.61e-16 ***
## ---
## Signif. codes:  0 '***' 0.001 '**' 0.01 '*' 0.05 '.' 0.1 ' ' 1
## 
## Residual standard error: 32.7 on 45 degrees of freedom
## Multiple R-squared:  0.8246, Adjusted R-squared:  0.809 
## F-statistic: 52.88 on 4 and 45 DF,  p-value: < 2.2e-16
\end{verbatim}

\begin{Shaded}
\begin{Highlighting}[]
\CommentTok{\# method 2 {-} ANOVAb}
\NormalTok{lm2\_anova }\OtherTok{\textless{}{-}} \FunctionTok{anova}\NormalTok{(sat\_lm1, sat\_lm2)}
\NormalTok{lm2\_anova}
\end{Highlighting}
\end{Shaded}

\begin{verbatim}
## Analysis of Variance Table
## 
## Model 1: total ~ expend + ratio + salary
## Model 2: total ~ expend + ratio + salary + takers
##   Res.Df    RSS Df Sum of Sq      F    Pr(>F)    
## 1     46 216812                                  
## 2     45  48124  1    168688 157.74 2.607e-16 ***
## ---
## Signif. codes:  0 '***' 0.001 '**' 0.01 '*' 0.05 '.' 0.1 ' ' 1
\end{verbatim}

\begin{Shaded}
\begin{Highlighting}[]
\CommentTok{\# verification}
\NormalTok{(lm2\_sum}\SpecialCharTok{$}\NormalTok{coefficients[}\StringTok{"takers"}\NormalTok{, }\StringTok{"t value"}\NormalTok{])}\SpecialCharTok{\^{}}\DecValTok{2} \SpecialCharTok{{-}}\NormalTok{ lm2\_anova}\SpecialCharTok{$}\StringTok{\textasciigrave{}}\AttributeTok{F}\StringTok{\textasciigrave{}}
\end{Highlighting}
\end{Shaded}

\begin{verbatim}
## [1]           NA 2.273737e-13
\end{verbatim}

\begin{Shaded}
\begin{Highlighting}[]
\NormalTok{(lm2\_sum}\SpecialCharTok{$}\NormalTok{coefficients[}\StringTok{"takers"}\NormalTok{, }\StringTok{"Pr(\textgreater{}|t|)"}\NormalTok{]) }\SpecialCharTok{{-}}\NormalTok{ lm2\_anova}\SpecialCharTok{$}\StringTok{\textasciigrave{}}\AttributeTok{Pr(\textgreater{}F)}\StringTok{\textasciigrave{}}
\end{Highlighting}
\end{Shaded}

\begin{verbatim}
## [1]            NA -7.494179e-30
\end{verbatim}

\hypertarget{regression-model-diagnostics}{%
\subsection{Regression Model
Diagnostics}\label{regression-model-diagnostics}}

\hypertarget{constant-variance-linearity-of-relationship}{%
\paragraph{Constant Variance \& Linearity of
Relationship}\label{constant-variance-linearity-of-relationship}}

We conduct diagnostics of our model \texttt{sat\_lm2}. First, we'll
check the Constant Variance assumption by plotting the residuals versus
the fitted y-values.

While the range over which the residuals vary is about equal on the left
and right hand sides, the distribution of points in the middle is skewed
to the negative side of the range. The smoothed curve suggest adding a
quadratic term to the model may be an appropriate transformation.

There are also 3 outliers noted on the plot - the values for North
Dakota, New Hampshire, and West Virginia need to be reviewed for
validity and may factor in additional transformations to the model.

Viewing the Scale-Location plot backs up our understanding of the model
- the residuals have a constant spread but we have a problem in the
shape of the model and thus the linearity of the relationship between
predictors and response. We also see Utah surface as another outlier.

\begin{Shaded}
\begin{Highlighting}[]
\FunctionTok{plot}\NormalTok{(sat\_lm2, }\DecValTok{1}\NormalTok{)}
\end{Highlighting}
\end{Shaded}

\includegraphics{TFsat_files/figure-latex/unnamed-chunk-6-1.pdf}

\begin{Shaded}
\begin{Highlighting}[]
\FunctionTok{plot}\NormalTok{(sat\_lm2, }\DecValTok{3}\NormalTok{)}
\end{Highlighting}
\end{Shaded}

\includegraphics{TFsat_files/figure-latex/unnamed-chunk-6-2.pdf}

\begin{Shaded}
\begin{Highlighting}[]
\CommentTok{\#ggplot(sat\_lm2, aes(x=fitted(sat\_lm2), y=residuals(sat\_lm2))) + geom\_jitter() + geom\_hline(yintercept=0, color="blue") + xlab("Fitted") + ylab("Residuals")}
\end{Highlighting}
\end{Shaded}

\hypertarget{normality-of-residuals}{%
\paragraph{Normality of Residuals}\label{normality-of-residuals}}

Next, we'll check for the normality of our distributed residuals. Using
a quantile - quantile plot and a histogram of residuals, we can see that
the shape of the distribution is approximately normal with some slight
deviation from normality at the tails, accounted for by our known
outliers plus another identified in the plot - Utah.

Notably, our data point for West Virginia seems to skew the distribution
of residuals slightly leftward in the histogram visualization. It's time
we take a look at the influence of each of these outliers on the model
as a whole.

\begin{Shaded}
\begin{Highlighting}[]
\FunctionTok{plot}\NormalTok{(sat\_lm2, }\DecValTok{2}\NormalTok{)}
\end{Highlighting}
\end{Shaded}

\includegraphics{TFsat_files/figure-latex/unnamed-chunk-7-1.pdf}

\begin{Shaded}
\begin{Highlighting}[]
\FunctionTok{ggplot}\NormalTok{(}\AttributeTok{data=}\NormalTok{sat\_lm2, }\FunctionTok{aes}\NormalTok{(}\AttributeTok{x=}\FunctionTok{residuals}\NormalTok{(sat\_lm2))) }\SpecialCharTok{+} \FunctionTok{geom\_histogram}\NormalTok{(}\AttributeTok{bins=}\DecValTok{15}\NormalTok{) }\SpecialCharTok{+} \FunctionTok{xlab}\NormalTok{(}\StringTok{"Residual"}\NormalTok{) }\SpecialCharTok{+} \FunctionTok{ylab}\NormalTok{(}\StringTok{"Frequency"}\NormalTok{)}
\end{Highlighting}
\end{Shaded}

\includegraphics{TFsat_files/figure-latex/unnamed-chunk-7-2.pdf}

\hypertarget{assessing-leverage}{%
\paragraph{Assessing leverage}\label{assessing-leverage}}

Looking at the residuals vs.~leverage plot below, we see that for the
most part, our data including the points of high leverage cluster within
the area between -2 and 2 standardized residuals.

The point of most concern is Utah, which exerts the most influence over
the dataset and falls just within the acceptable range of Cook's
Distance (0.5). This is not enough of an anomaly to impact our
assumptions of the model's validity.

\begin{Shaded}
\begin{Highlighting}[]
\FunctionTok{plot}\NormalTok{(sat\_lm2, }\DecValTok{5}\NormalTok{)}
\end{Highlighting}
\end{Shaded}

\includegraphics{TFsat_files/figure-latex/unnamed-chunk-8-1.pdf}

\begin{Shaded}
\begin{Highlighting}[]
\CommentTok{\#ggplot(data=sat\_lm2, aes(x=residuals(sat\_lm2))) + geom\_histogram(bins=15) + xlab("Residual") + ylab("Frequency")}
\end{Highlighting}
\end{Shaded}


\end{document}
