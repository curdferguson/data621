% Options for packages loaded elsewhere
\PassOptionsToPackage{unicode}{hyperref}
\PassOptionsToPackage{hyphens}{url}
%
\documentclass[
]{article}
\usepackage{amsmath,amssymb}
\usepackage{lmodern}
\usepackage{ifxetex,ifluatex}
\ifnum 0\ifxetex 1\fi\ifluatex 1\fi=0 % if pdftex
  \usepackage[T1]{fontenc}
  \usepackage[utf8]{inputenc}
  \usepackage{textcomp} % provide euro and other symbols
\else % if luatex or xetex
  \usepackage{unicode-math}
  \defaultfontfeatures{Scale=MatchLowercase}
  \defaultfontfeatures[\rmfamily]{Ligatures=TeX,Scale=1}
\fi
% Use upquote if available, for straight quotes in verbatim environments
\IfFileExists{upquote.sty}{\usepackage{upquote}}{}
\IfFileExists{microtype.sty}{% use microtype if available
  \usepackage[]{microtype}
  \UseMicrotypeSet[protrusion]{basicmath} % disable protrusion for tt fonts
}{}
\makeatletter
\@ifundefined{KOMAClassName}{% if non-KOMA class
  \IfFileExists{parskip.sty}{%
    \usepackage{parskip}
  }{% else
    \setlength{\parindent}{0pt}
    \setlength{\parskip}{6pt plus 2pt minus 1pt}}
}{% if KOMA class
  \KOMAoptions{parskip=half}}
\makeatother
\usepackage{xcolor}
\IfFileExists{xurl.sty}{\usepackage{xurl}}{} % add URL line breaks if available
\IfFileExists{bookmark.sty}{\usepackage{bookmark}}{\usepackage{hyperref}}
\hypersetup{
  pdftitle={Teen Gambling in Britain},
  pdfauthor={Tyler Frankenberg},
  hidelinks,
  pdfcreator={LaTeX via pandoc}}
\urlstyle{same} % disable monospaced font for URLs
\usepackage[margin=1in]{geometry}
\usepackage{color}
\usepackage{fancyvrb}
\newcommand{\VerbBar}{|}
\newcommand{\VERB}{\Verb[commandchars=\\\{\}]}
\DefineVerbatimEnvironment{Highlighting}{Verbatim}{commandchars=\\\{\}}
% Add ',fontsize=\small' for more characters per line
\usepackage{framed}
\definecolor{shadecolor}{RGB}{248,248,248}
\newenvironment{Shaded}{\begin{snugshade}}{\end{snugshade}}
\newcommand{\AlertTok}[1]{\textcolor[rgb]{0.94,0.16,0.16}{#1}}
\newcommand{\AnnotationTok}[1]{\textcolor[rgb]{0.56,0.35,0.01}{\textbf{\textit{#1}}}}
\newcommand{\AttributeTok}[1]{\textcolor[rgb]{0.77,0.63,0.00}{#1}}
\newcommand{\BaseNTok}[1]{\textcolor[rgb]{0.00,0.00,0.81}{#1}}
\newcommand{\BuiltInTok}[1]{#1}
\newcommand{\CharTok}[1]{\textcolor[rgb]{0.31,0.60,0.02}{#1}}
\newcommand{\CommentTok}[1]{\textcolor[rgb]{0.56,0.35,0.01}{\textit{#1}}}
\newcommand{\CommentVarTok}[1]{\textcolor[rgb]{0.56,0.35,0.01}{\textbf{\textit{#1}}}}
\newcommand{\ConstantTok}[1]{\textcolor[rgb]{0.00,0.00,0.00}{#1}}
\newcommand{\ControlFlowTok}[1]{\textcolor[rgb]{0.13,0.29,0.53}{\textbf{#1}}}
\newcommand{\DataTypeTok}[1]{\textcolor[rgb]{0.13,0.29,0.53}{#1}}
\newcommand{\DecValTok}[1]{\textcolor[rgb]{0.00,0.00,0.81}{#1}}
\newcommand{\DocumentationTok}[1]{\textcolor[rgb]{0.56,0.35,0.01}{\textbf{\textit{#1}}}}
\newcommand{\ErrorTok}[1]{\textcolor[rgb]{0.64,0.00,0.00}{\textbf{#1}}}
\newcommand{\ExtensionTok}[1]{#1}
\newcommand{\FloatTok}[1]{\textcolor[rgb]{0.00,0.00,0.81}{#1}}
\newcommand{\FunctionTok}[1]{\textcolor[rgb]{0.00,0.00,0.00}{#1}}
\newcommand{\ImportTok}[1]{#1}
\newcommand{\InformationTok}[1]{\textcolor[rgb]{0.56,0.35,0.01}{\textbf{\textit{#1}}}}
\newcommand{\KeywordTok}[1]{\textcolor[rgb]{0.13,0.29,0.53}{\textbf{#1}}}
\newcommand{\NormalTok}[1]{#1}
\newcommand{\OperatorTok}[1]{\textcolor[rgb]{0.81,0.36,0.00}{\textbf{#1}}}
\newcommand{\OtherTok}[1]{\textcolor[rgb]{0.56,0.35,0.01}{#1}}
\newcommand{\PreprocessorTok}[1]{\textcolor[rgb]{0.56,0.35,0.01}{\textit{#1}}}
\newcommand{\RegionMarkerTok}[1]{#1}
\newcommand{\SpecialCharTok}[1]{\textcolor[rgb]{0.00,0.00,0.00}{#1}}
\newcommand{\SpecialStringTok}[1]{\textcolor[rgb]{0.31,0.60,0.02}{#1}}
\newcommand{\StringTok}[1]{\textcolor[rgb]{0.31,0.60,0.02}{#1}}
\newcommand{\VariableTok}[1]{\textcolor[rgb]{0.00,0.00,0.00}{#1}}
\newcommand{\VerbatimStringTok}[1]{\textcolor[rgb]{0.31,0.60,0.02}{#1}}
\newcommand{\WarningTok}[1]{\textcolor[rgb]{0.56,0.35,0.01}{\textbf{\textit{#1}}}}
\usepackage{graphicx}
\makeatletter
\def\maxwidth{\ifdim\Gin@nat@width>\linewidth\linewidth\else\Gin@nat@width\fi}
\def\maxheight{\ifdim\Gin@nat@height>\textheight\textheight\else\Gin@nat@height\fi}
\makeatother
% Scale images if necessary, so that they will not overflow the page
% margins by default, and it is still possible to overwrite the defaults
% using explicit options in \includegraphics[width, height, ...]{}
\setkeys{Gin}{width=\maxwidth,height=\maxheight,keepaspectratio}
% Set default figure placement to htbp
\makeatletter
\def\fps@figure{htbp}
\makeatother
\setlength{\emergencystretch}{3em} % prevent overfull lines
\providecommand{\tightlist}{%
  \setlength{\itemsep}{0pt}\setlength{\parskip}{0pt}}
\setcounter{secnumdepth}{-\maxdimen} % remove section numbering
\ifluatex
  \usepackage{selnolig}  % disable illegal ligatures
\fi

\title{Teen Gambling in Britain}
\author{Tyler Frankenberg}
\date{1/30/2022}

\begin{document}
\maketitle

\hypertarget{import-packages}{%
\subsection{Import packages}\label{import-packages}}

\begin{Shaded}
\begin{Highlighting}[]
\FunctionTok{library}\NormalTok{(tidyverse)}
\end{Highlighting}
\end{Shaded}

\hypertarget{import-data-basic-cleaning-transformations}{%
\subsection{Import data \& basic cleaning/
transformations}\label{import-data-basic-cleaning-transformations}}

The \texttt{teengamb} dataset from package \texttt{faraway} includes
data from a 1988 survey on teenage gambling in Britain. Per
\emph{RDocumentation}, it includes the following columns:

\begin{itemize}
\tightlist
\item
  \texttt{sex}: 0=male, 1=female
\item
  \texttt{status}: socioeconomic status score based on parents'
  occupation\\
\item
  \texttt{income}: weekly income in GBP
\item
  \texttt{verbal}: verbal score of the number of words out of 12
  correctly defined
\item
  \texttt{gamble}: gambling activity (weighted score combining annual
  gambling frequency and amount wagered)
\end{itemize}

The data was collected via a survey conducted in two classrooms of 13-14
year olds in Exeter, England

\emph{Sources:
\url{https://www.rdocumentation.org/packages/faraway/versions/1.0.7/topics/}\\
\url{https://www.researchgate.net/publication/226877934_Gambling_in_young_adolescents}}

\hypertarget{dataset-transformations}{%
\subsubsection{Dataset transformations}\label{dataset-transformations}}

We're going to do some simple transformations on this dataset.\\
- first, we'll replace the binary values of sex with their corresponding
label; either ``male'' or ``female''\\
- second, we'll change this column's type to factor\\
- finally, we'll select only the columns we need for the analysis

\begin{Shaded}
\begin{Highlighting}[]
\NormalTok{url }\OtherTok{\textless{}{-}} \StringTok{"https://raw.githubusercontent.com/curdferguson/data621/main/datasets/teengamb.txt"}

\NormalTok{teengamb }\OtherTok{\textless{}{-}} \FunctionTok{read\_tsv}\NormalTok{(url, }\AttributeTok{skip =} \DecValTok{1}\NormalTok{, }\AttributeTok{col\_names =} \FunctionTok{c}\NormalTok{(}\StringTok{"index"}\NormalTok{, }\StringTok{"sex1"}\NormalTok{, }\StringTok{"status"}\NormalTok{, }\StringTok{"income"}\NormalTok{, }\StringTok{"verbal"}\NormalTok{, }\StringTok{"gamble"}\NormalTok{), }\AttributeTok{show\_col\_types=}\ConstantTok{FALSE}\NormalTok{)}

\NormalTok{teengamb }\OtherTok{\textless{}{-}}\NormalTok{ teengamb }\SpecialCharTok{\%\textgreater{}\%} 
    \FunctionTok{mutate}\NormalTok{(}\AttributeTok{sex =} \FunctionTok{case\_when}\NormalTok{(}
\NormalTok{      teengamb}\SpecialCharTok{$}\NormalTok{sex1 }\SpecialCharTok{==} \DecValTok{0} \SpecialCharTok{\textasciitilde{}} \StringTok{"male"}\NormalTok{,}
\NormalTok{      teengamb}\SpecialCharTok{$}\NormalTok{sex1 }\SpecialCharTok{==} \DecValTok{1} \SpecialCharTok{\textasciitilde{}} \StringTok{"female"}\NormalTok{))}

\NormalTok{teengamb}\SpecialCharTok{$}\NormalTok{sex }\OtherTok{\textless{}{-}}\NormalTok{ teengamb}\SpecialCharTok{$}\NormalTok{sex }\SpecialCharTok{\%\textgreater{}\%} \FunctionTok{as\_factor}\NormalTok{()}

\NormalTok{teengamb }\OtherTok{\textless{}{-}}\NormalTok{ teengamb[,}\DecValTok{3}\SpecialCharTok{:}\DecValTok{7}\NormalTok{]}
\end{Highlighting}
\end{Shaded}

\hypertarget{glimpse-dataset-structure-and-each-columns-summary-statistics}{%
\subsection{Glimpse dataset structure and each column's summary
statistics}\label{glimpse-dataset-structure-and-each-columns-summary-statistics}}

We'll glimpse the dataset's structure by viewing its first five rows, as
well as a brief summary of each column's distribution, and histograms
for each of the numerical datasets.

The most important finding from our initial analysis is that the
\texttt{gamble} variable - which is going to be the target in our linear
model - is heavily right-skewed; it seems to follow an exponential
distribution.

We'll correct for this in our linear model by creating an additional
column with the base-10 logarithm of \texttt{gamble}.

\begin{Shaded}
\begin{Highlighting}[]
\NormalTok{teengamb }\SpecialCharTok{\%\textgreater{}\%} \FunctionTok{head}\NormalTok{(}\DecValTok{5}\NormalTok{)}
\end{Highlighting}
\end{Shaded}

\begin{verbatim}
## # A tibble: 5 x 5
##   status income verbal gamble sex   
##    <dbl>  <dbl>  <dbl>  <dbl> <fct> 
## 1     51    2        8    0   female
## 2     28    2.5      8    0   female
## 3     37    2        6    0   female
## 4     28    7        4    7.3 female
## 5     65    2        8   19.6 female
\end{verbatim}

\begin{Shaded}
\begin{Highlighting}[]
\NormalTok{teengamb }\SpecialCharTok{\%\textgreater{}\%} \FunctionTok{summary}\NormalTok{()}
\end{Highlighting}
\end{Shaded}

\begin{verbatim}
##      status          income           verbal          gamble          sex    
##  Min.   :18.00   Min.   : 0.600   Min.   : 1.00   Min.   :  0.0   female:19  
##  1st Qu.:28.00   1st Qu.: 2.000   1st Qu.: 6.00   1st Qu.:  1.1   male  :28  
##  Median :43.00   Median : 3.250   Median : 7.00   Median :  6.0              
##  Mean   :45.23   Mean   : 4.642   Mean   : 6.66   Mean   : 19.3              
##  3rd Qu.:61.50   3rd Qu.: 6.210   3rd Qu.: 8.00   3rd Qu.: 19.4              
##  Max.   :75.00   Max.   :15.000   Max.   :10.00   Max.   :156.0
\end{verbatim}

\begin{Shaded}
\begin{Highlighting}[]
\NormalTok{cols\_list }\OtherTok{\textless{}{-}} \FunctionTok{colnames}\NormalTok{(teengamb)}

\NormalTok{teengamb\_numeric }\OtherTok{\textless{}{-}}\NormalTok{ teengamb }\SpecialCharTok{\%\textgreater{}\%} \FunctionTok{select}\NormalTok{(}\FunctionTok{where}\NormalTok{(is.numeric))}
\NormalTok{teengamb\_numeric\_long }\OtherTok{\textless{}{-}}\NormalTok{ teengamb\_numeric }\SpecialCharTok{\%\textgreater{}\%} \FunctionTok{pivot\_longer}\NormalTok{(}\FunctionTok{colnames}\NormalTok{(teengamb\_numeric)) }\SpecialCharTok{\%\textgreater{}\%} \FunctionTok{as.data.frame}\NormalTok{()}

\FunctionTok{ggplot}\NormalTok{(}\AttributeTok{data=}\NormalTok{teengamb\_numeric\_long, }\FunctionTok{aes}\NormalTok{(}\AttributeTok{x=}\NormalTok{value)) }\SpecialCharTok{+} 
  \FunctionTok{geom\_histogram}\NormalTok{(}\FunctionTok{aes}\NormalTok{(}\AttributeTok{y=}\NormalTok{..density..), }\AttributeTok{bins=}\DecValTok{20}\NormalTok{) }\SpecialCharTok{+} 
  \FunctionTok{facet\_wrap}\NormalTok{(}\SpecialCharTok{\textasciitilde{}}\NormalTok{ name, }\AttributeTok{scales =} \StringTok{"free"}\NormalTok{)}
\end{Highlighting}
\end{Shaded}

\includegraphics{TFteengamb_files/figure-latex/unnamed-chunk-3-1.pdf}

\hypertarget{apply-log-transformation-to-gamble}{%
\subsection{\texorpdfstring{Apply log transformation to
\texttt{gamble}}{Apply log transformation to gamble}}\label{apply-log-transformation-to-gamble}}

Since the base-10 logarithm of 0 results in a value of negative 1, which
we won't be able to pass to our \texttt{lm} function, we're going to
apply an additional transformation to allow for compatibility with the
log transformation. We will add 1 to each of the values in
\texttt{teengamb\$gamble} by 10 before taking the logarithm.

\begin{Shaded}
\begin{Highlighting}[]
\NormalTok{teengamb }\OtherTok{\textless{}{-}}\NormalTok{ teengamb }\SpecialCharTok{\%\textgreater{}\%} \FunctionTok{mutate}\NormalTok{(}\AttributeTok{log\_gamble =} \FunctionTok{log10}\NormalTok{((teengamb}\SpecialCharTok{$}\NormalTok{gamble }\SpecialCharTok{+} \DecValTok{1}\NormalTok{)))}

\FunctionTok{ggplot}\NormalTok{(}\AttributeTok{data=}\NormalTok{teengamb, }\FunctionTok{aes}\NormalTok{(}\AttributeTok{x=}\NormalTok{log\_gamble)) }\SpecialCharTok{+} 
  \FunctionTok{geom\_histogram}\NormalTok{(}\FunctionTok{aes}\NormalTok{(}\AttributeTok{y=}\NormalTok{..density..), }\AttributeTok{bins=}\DecValTok{20}\NormalTok{)}
\end{Highlighting}
\end{Shaded}

\includegraphics{TFteengamb_files/figure-latex/unnamed-chunk-4-1.pdf}

\hypertarget{view-pairs}{%
\subsection{View pairs}\label{view-pairs}}

Having transformed our variable, let's use \texttt{pairs()} to view the
relationsihps between each of the variables.

We can see that \texttt{income} may have a linear relationship with
\texttt{log\_gamble}, while \texttt{verbal} and \texttt{status} are
somewhat more muddled.

\begin{Shaded}
\begin{Highlighting}[]
\FunctionTok{pairs}\NormalTok{(teengamb)}
\end{Highlighting}
\end{Shaded}

\includegraphics{TFteengamb_files/figure-latex/unnamed-chunk-5-1.pdf}

\hypertarget{view-pairs-view-individual-relationships-separated-by-sex}{%
\subsection{View pairs, view individual relationships separated by
sex}\label{view-pairs-view-individual-relationships-separated-by-sex}}

It's also clear from our \texttt{pairs()} output that there's a
noticeable correlation between values with \texttt{sex\ ==\ male} and
\texttt{log\_gamble}.

Let's view the relationships with our numerical predictors, with a fill
color indicating the value of \texttt{sex}.

\begin{Shaded}
\begin{Highlighting}[]
\NormalTok{cbPalette }\OtherTok{\textless{}{-}} \FunctionTok{c}\NormalTok{(}\StringTok{"\#999999"}\NormalTok{, }\StringTok{"\#E69F00"}\NormalTok{, }\StringTok{"\#56B4E9"}\NormalTok{, }\StringTok{"\#009E73"}\NormalTok{, }\StringTok{"\#F0E442"}\NormalTok{, }\StringTok{"\#0072B2"}\NormalTok{, }\StringTok{"\#D55E00"}\NormalTok{, }\StringTok{"\#CC79A7"}\NormalTok{)}

\CommentTok{\#This colorblind{-}friendly color palette is from http://jfly.iam.u{-}tokyo.ac.jp/color/.  Referencing code courtesy of http://www.cookbook{-}r.com/Graphs/Colors\_(ggplot2)/}

\CommentTok{\#ggplot(data=teengamb, aes(x=status, y=gamble, group=sex, fill = sex)) + }
  \CommentTok{\#geom\_jitter(aes(col=sex)) + scale\_colour\_manual(values=c("\#E69F00", "\#0072B2"))}

\FunctionTok{ggplot}\NormalTok{(}\AttributeTok{data=}\NormalTok{teengamb, }\FunctionTok{aes}\NormalTok{(}\AttributeTok{x=}\NormalTok{status, }\AttributeTok{y=}\NormalTok{log\_gamble, }\AttributeTok{group=}\NormalTok{sex, }\AttributeTok{fill =}\NormalTok{ sex)) }\SpecialCharTok{+} \FunctionTok{geom\_jitter}\NormalTok{(}\FunctionTok{aes}\NormalTok{(}\AttributeTok{col=}\NormalTok{sex)) }\SpecialCharTok{+} \FunctionTok{scale\_colour\_manual}\NormalTok{(}\AttributeTok{values=}\FunctionTok{c}\NormalTok{(}\StringTok{"\#E69F00"}\NormalTok{, }\StringTok{"\#0072B2"}\NormalTok{))}
\end{Highlighting}
\end{Shaded}

\includegraphics{TFteengamb_files/figure-latex/unnamed-chunk-6-1.pdf}

\begin{Shaded}
\begin{Highlighting}[]
\CommentTok{\#ggplot(data=teengamb, aes(x=income, y=gamble, group=sex, fill = sex)) + }
  \CommentTok{\#geom\_jitter(aes(col=sex)) + scale\_colour\_manual(values=c("\#E69F00", "\#0072B2"))}

\FunctionTok{ggplot}\NormalTok{(}\AttributeTok{data=}\NormalTok{teengamb, }\FunctionTok{aes}\NormalTok{(}\AttributeTok{x=}\NormalTok{income, }\AttributeTok{y=}\NormalTok{log\_gamble, }\AttributeTok{group=}\NormalTok{sex, }\AttributeTok{fill =}\NormalTok{ sex)) }\SpecialCharTok{+} \FunctionTok{geom\_jitter}\NormalTok{(}\FunctionTok{aes}\NormalTok{(}\AttributeTok{col=}\NormalTok{sex)) }\SpecialCharTok{+} \FunctionTok{scale\_colour\_manual}\NormalTok{(}\AttributeTok{values=}\FunctionTok{c}\NormalTok{(}\StringTok{"\#E69F00"}\NormalTok{, }\StringTok{"\#0072B2"}\NormalTok{))}
\end{Highlighting}
\end{Shaded}

\includegraphics{TFteengamb_files/figure-latex/unnamed-chunk-6-2.pdf}

\begin{Shaded}
\begin{Highlighting}[]
\CommentTok{\#ggplot(data=teengamb, aes(x=verbal, y=gamble, group=sex, fill = sex)) + }
  \CommentTok{\#geom\_jitter(aes(col=sex)) + scale\_colour\_manual(values=c("\#E69F00", "\#0072B2"))}

\FunctionTok{ggplot}\NormalTok{(}\AttributeTok{data=}\NormalTok{teengamb, }\FunctionTok{aes}\NormalTok{(}\AttributeTok{x=}\NormalTok{verbal, }\AttributeTok{y=}\NormalTok{log\_gamble, }\AttributeTok{group=}\NormalTok{sex, }\AttributeTok{fill =}\NormalTok{ sex)) }\SpecialCharTok{+} \FunctionTok{geom\_jitter}\NormalTok{(}\FunctionTok{aes}\NormalTok{(}\AttributeTok{col=}\NormalTok{sex)) }\SpecialCharTok{+} \FunctionTok{scale\_colour\_manual}\NormalTok{(}\AttributeTok{values=}\FunctionTok{c}\NormalTok{(}\StringTok{"\#E69F00"}\NormalTok{, }\StringTok{"\#0072B2"}\NormalTok{))}
\end{Highlighting}
\end{Shaded}

\includegraphics{TFteengamb_files/figure-latex/unnamed-chunk-6-3.pdf}

\hypertarget{construct-a-linear-model-using-all-variables}{%
\subsection{Construct a linear model using all
variables}\label{construct-a-linear-model-using-all-variables}}

Starting with \texttt{sex}, \texttt{status}, \texttt{income}, and
\texttt{verbal} as our predictors, we'll fit a linear model to the data.

The distribution of residuals are not symmetrically distributed around
their median, and the standard error of the residual is not uniformly
close to 1.5 times the two quartiles, so our assumption of
normally-distributed residuals is likely violated.

Each of the coefficients are statistically significant at \(\alpha\) =
0.05, however it's concerning that the coefficients of \texttt{income},
\texttt{verbal}, \texttt{status}, and \texttt{sexmale} are not larger
with respect to their respective coefficient standard errors.

We'll want to tweak this model by removing one of the least significant
variables.

\begin{Shaded}
\begin{Highlighting}[]
\NormalTok{teengamb\_lm1\_log }\OtherTok{\textless{}{-}} \FunctionTok{lm}\NormalTok{(log\_gamble }\SpecialCharTok{\textasciitilde{}}\NormalTok{ sex }\SpecialCharTok{+}\NormalTok{ status }\SpecialCharTok{+}\NormalTok{ income }\SpecialCharTok{+}\NormalTok{ verbal, }\AttributeTok{data=}\NormalTok{teengamb)}
\FunctionTok{summary}\NormalTok{(teengamb\_lm1\_log, }\AttributeTok{cor=}\ConstantTok{TRUE}\NormalTok{)}
\end{Highlighting}
\end{Shaded}

\begin{verbatim}
## 
## Call:
## lm(formula = log_gamble ~ sex + status + income + verbal, data = teengamb)
## 
## Residuals:
##      Min       1Q   Median       3Q      Max 
## -1.02065 -0.24696  0.00179  0.31057  0.82655 
## 
## Coefficients:
##              Estimate Std. Error t value Pr(>|t|)    
## (Intercept)  0.366980   0.306436   1.198   0.2378    
## sexmale      0.378355   0.170539   2.219   0.0320 *  
## status       0.012955   0.005838   2.219   0.0320 *  
## income       0.093654   0.021297   4.398 7.33e-05 ***
## verbal      -0.113631   0.045114  -2.519   0.0157 *  
## ---
## Signif. codes:  0 '***' 0.001 '**' 0.01 '*' 0.05 '.' 0.1 ' ' 1
## 
## Residual standard error: 0.4713 on 42 degrees of freedom
## Multiple R-squared:  0.5206, Adjusted R-squared:  0.475 
## F-statistic:  11.4 on 4 and 42 DF,  p-value: 2.347e-06
## 
## Correlation of Coefficients:
##         (Intercept) sexmale status income
## sexmale  0.04                            
## status  -0.27       -0.55                
## income  -0.49       -0.29    0.34        
## verbal  -0.58        0.20   -0.53  -0.02
\end{verbatim}

\hypertarget{construct-linear-model---backward-elimination---remove-status}{%
\subsection{\texorpdfstring{Construct linear model - backward
elimination - remove
\texttt{status}}{Construct linear model - backward elimination - remove status}}\label{construct-linear-model---backward-elimination---remove-status}}

Using backward elimination, we'll remove \texttt{status} from our model.
Our adjusted \(R^2\) has diminished somewhat, but our Residuals are now
more normally distributed about their mean.

However, the impact of \texttt{verbal} is now no longer statistically
significant, so we'll remove it in the next step.

\begin{Shaded}
\begin{Highlighting}[]
\NormalTok{teengamb\_lm2\_log }\OtherTok{\textless{}{-}} \FunctionTok{lm}\NormalTok{(log\_gamble }\SpecialCharTok{\textasciitilde{}}\NormalTok{ sex }\SpecialCharTok{+}\NormalTok{ income }\SpecialCharTok{+}\NormalTok{ verbal, }\AttributeTok{data=}\NormalTok{teengamb)}
\FunctionTok{summary}\NormalTok{(teengamb\_lm2\_log, }\AttributeTok{cor=}\ConstantTok{TRUE}\NormalTok{)}
\end{Highlighting}
\end{Shaded}

\begin{verbatim}
## 
## Call:
## lm(formula = log_gamble ~ sex + income + verbal, data = teengamb)
## 
## Residuals:
##      Min       1Q   Median       3Q      Max 
## -1.06455 -0.33920 -0.00057  0.36434  1.09444 
## 
## Coefficients:
##             Estimate Std. Error t value Pr(>|t|)    
## (Intercept)  0.55087    0.30818   1.787 0.080909 .  
## sexmale      0.58716    0.14857   3.952 0.000284 ***
## income       0.07781    0.02096   3.712 0.000586 ***
## verbal      -0.06088    0.04005  -1.520 0.135825    
## ---
## Signif. codes:  0 '***' 0.001 '**' 0.01 '*' 0.05 '.' 0.1 ' ' 1
## 
## Residual standard error: 0.4923 on 43 degrees of freedom
## Multiple R-squared:  0.4645, Adjusted R-squared:  0.4271 
## F-statistic: 12.43 on 3 and 43 DF,  p-value: 5.488e-06
## 
## Correlation of Coefficients:
##         (Intercept) sexmale income
## sexmale -0.13                     
## income  -0.44       -0.14         
## verbal  -0.89       -0.13    0.19
\end{verbatim}

\hypertarget{construct-linear-model---backward-elimination---remove-verbal}{%
\subsection{\texorpdfstring{Construct linear model - backward
elimination - remove
\texttt{verbal}}{Construct linear model - backward elimination - remove verbal}}\label{construct-linear-model---backward-elimination---remove-verbal}}

Removing \texttt{verbal} from the model yields a new one with
symmetrically distributed residuals about their mean, and a residual
standard error close to 1.5 times the 1st and 3rd Quartile residuals`
absolute distance from the mean.

The two coefficients are statistically significant; however, their
values are only 3-4 times their respective standard errors, which means
there is likely too much error to make reliable predictions.

This may be a function of too few survey responses.

As a side note, We should also question whether the responses can be
considered truly independent of one another, given they were obtained
from two classrooms in a school where students knew each other (and
likely gambled together), rather than from a representative sample of
the population.

\begin{Shaded}
\begin{Highlighting}[]
\NormalTok{teengamb\_lm3\_log }\OtherTok{\textless{}{-}} \FunctionTok{lm}\NormalTok{(log\_gamble }\SpecialCharTok{\textasciitilde{}}\NormalTok{ sex }\SpecialCharTok{+}\NormalTok{ income, }\AttributeTok{data=}\NormalTok{teengamb)}
\FunctionTok{summary}\NormalTok{(teengamb\_lm3\_log, }\AttributeTok{cor=}\ConstantTok{TRUE}\NormalTok{)}
\end{Highlighting}
\end{Shaded}

\begin{verbatim}
## 
## Call:
## lm(formula = log_gamble ~ sex + income, data = teengamb)
## 
## Residuals:
##      Min       1Q   Median       3Q      Max 
## -0.98610 -0.33573  0.03733  0.36488  1.01134 
## 
## Coefficients:
##             Estimate Std. Error t value Pr(>|t|)    
## (Intercept)  0.13479    0.14367   0.938 0.353298    
## sexmale      0.55777    0.14949   3.731 0.000543 ***
## income       0.08387    0.02088   4.017 0.000227 ***
## ---
## Signif. codes:  0 '***' 0.001 '**' 0.01 '*' 0.05 '.' 0.1 ' ' 1
## 
## Residual standard error: 0.4996 on 44 degrees of freedom
## Multiple R-squared:  0.4357, Adjusted R-squared:   0.41 
## F-statistic: 16.98 on 2 and 44 DF,  p-value: 3.416e-06
## 
## Correlation of Coefficients:
##         (Intercept) sexmale
## sexmale -0.54              
## income  -0.60       -0.12
\end{verbatim}

\hypertarget{remove-log-transformation-and-interpret-results}{%
\subsection{Remove log transformation and interpret
results\ldots{}}\label{remove-log-transformation-and-interpret-results}}

Even though the model is likely too problematic to make reliable
predictions, it's worth going through the process of ``de-logging'' in
order to demonstrate interpreting log-transformed results.

In the case of this model, where the response variable has been
log-transformed but the predictor variables have not, we'll ``de-log''
by exponentiating each of the coefficients, subtracting 1, and
multiplying by 100 to get the ``Percentage impact'' on the
(untransformed) response variable.

\begin{Shaded}
\begin{Highlighting}[]
\NormalTok{lm3\_sexmale\_est }\OtherTok{\textless{}{-}}\NormalTok{ teengamb\_lm3\_log}\SpecialCharTok{$}\NormalTok{coefficients[[}\StringTok{"sexmale"}\NormalTok{]]}
\NormalTok{lm3\_sexmale\_delogged }\OtherTok{\textless{}{-}}\NormalTok{ (}\FunctionTok{exp}\NormalTok{(lm3\_sexmale\_est) }\SpecialCharTok{{-}} \DecValTok{1}\NormalTok{)}\SpecialCharTok{*}\DecValTok{100}

\NormalTok{lm3\_income\_est }\OtherTok{\textless{}{-}}\NormalTok{ teengamb\_lm3\_log}\SpecialCharTok{$}\NormalTok{coefficients[[}\StringTok{"income"}\NormalTok{]]}
\NormalTok{lm3\_income\_delogged }\OtherTok{\textless{}{-}}\NormalTok{ (}\FunctionTok{exp}\NormalTok{(lm3\_income\_est) }\SpecialCharTok{{-}} \DecValTok{1}\NormalTok{)}\SpecialCharTok{*}\DecValTok{100}

\FunctionTok{print}\NormalTok{(}\FunctionTok{paste0}\NormalTok{(}\StringTok{"A value of \textquotesingle{}male\textquotesingle{} for variable \textasciigrave{}sex\textasciigrave{} increases the value of \textasciigrave{}gamble\textasciigrave{} by "}\NormalTok{, }\FunctionTok{round}\NormalTok{(lm3\_sexmale\_delogged, }\DecValTok{3}\NormalTok{), }\StringTok{"\%."}\NormalTok{))}
\end{Highlighting}
\end{Shaded}

\begin{verbatim}
## [1] "A value of 'male' for variable `sex` increases the value of `gamble` by 74.678%."
\end{verbatim}

\begin{Shaded}
\begin{Highlighting}[]
\FunctionTok{cat}\NormalTok{(}\StringTok{"}\SpecialCharTok{\textbackslash{}n}\StringTok{"}\NormalTok{)}
\end{Highlighting}
\end{Shaded}

\begin{Shaded}
\begin{Highlighting}[]
\FunctionTok{print}\NormalTok{(}\FunctionTok{paste0}\NormalTok{(}\StringTok{"Each 1 GBP increase in \textasciigrave{}income\textasciigrave{} increases the value of \textasciigrave{}gamble\textasciigrave{} by "}\NormalTok{, }\FunctionTok{round}\NormalTok{(lm3\_income\_delogged, }\DecValTok{3}\NormalTok{), }\StringTok{"\%."}\NormalTok{))}
\end{Highlighting}
\end{Shaded}

\begin{verbatim}
## [1] "Each 1 GBP increase in `income` increases the value of `gamble` by 8.749%."
\end{verbatim}

\end{document}
